\documentclass[10pt,a4paper]{article}
\usepackage[utf8]{inputenc}
\usepackage[french]{babel}
\usepackage[T1]{fontenc}
\usepackage{amsmath,amsfonts,amssymb}
\usepackage[squaren, Gray, cdot]{SIunits}
\usepackage{tikz}
\usepackage{circuitikz}
\usepackage[left=2cm,right=2cm,top=2cm,bottom=2cm]{geometry}


\tikzset{
xmin/.store in=\xmin, xmin/.default=-3, xmin=-3,
xmax/.store in=\xmax, xmax/.default=3, xmax=3,
ymin/.store in=\ymin, ymin/.default=-3, ymin=-3,
ymax/.store in=\ymax, ymax/.default=3, ymax=3,
}
% Commande qui trace la grille entre (xmin,ymin) et (xmax,ymax)
\newcommand {\grille}
{\draw[help lines] (\xmin,\ymin) grid (\xmax,\ymax);}
% Commande \axes
\newcommand {\axes} {
\draw[->] (\xmin,0) -- (\xmax,0);
\draw[->] (0,\ymin) -- (0,\ymax);
}
% Commande qui limite l'affichage à (xmin,ymin) et (xmax,ymax)
\newcommand {\fenetre}
{\clip (\xmin,\ymin) rectangle (\xmax,\ymax);} 
 
\begin{document}{\bf Cours d'electronique} 

\section{Circuit série et U=RI}
Dans un circuit la tension (U exprimée en \volt ) est égale à la résistance (R exprimée en \ohm) multiplié par l'intensité (I exprimée en \ampere) ce qui donne l'équation~\eqref{uri}
\begin{equation}
U=R*I
\label{uri}
\end{equation}

Remarque en série les résistances s'additionnent.

Prenons un circuit avec les résistances R1= 5 \ohm, R2= 20 \ohm, R3= 30 \ohm
\shorthandoff{:!}
\begin{figure}[h!]
  \begin{center}
    \begin{circuitikz}
      \draw (0,0)
      to[V,v=$U_q$] (0,2) % The voltage source
      to[R=$R_1$,v_<=$U_1$] (2,2) % R1
      to[R=$R_2$,v_<=$U_2$] (4,2) % R2
      to[R=$R_3$,v_<=$U_3$] (6,2) % R3
      to[short,i=$i$] (6,0)
      to[short] (0,0);
    \end{circuitikz}
    \caption{Circuit simple.}
  \end{center}
\end{figure}

Remarque en série les résistances s'additionnent.
\begin{align}
R_t &=R_1+R_2+R_3 \\
    &=5+20+30 \\
    &=55 \ohm
\end{align}

\begin{align}
I&=\frac{U}{R} \\
I&=\frac{12}{55} \\
I&=0,218 \ampere \\
\end{align}
La loi d'additivité des tensions se vérifie donc 
\begin{align}
U1&=5*0,218=1,09 \volt \\
U2&=20*0,218=4,36 \volt \\
U3&=30*0,218=6,54 \volt \\
U_q&=U1+U2+U3~=12 \volt
\end{align}

\pagebreak 

\section{Circuit Parellèle et Conductance}
La conductance est l'inverse de la résistance $ C= \frac{1}{Rt} $

\shorthandoff{:!}
\begin{figure}[h!]
  \begin{center}
    \begin{circuitikz}
      \draw (0,0)
      to[V,v=$U_q$] (0,5) % The voltage source
      to[short] (4,5)
      to[short,i_=$i_t$] (4,4)
      (4,4) to[short,i_=$i_1$] (3,4)
      to[short] (3,3)
      (4,4) to[short,i_=$i_2$] (5,4)
      to[short] (5,3)
      (3,3) to[R=$R_1$,v_<=$U_1$] (3,2) % R1
      to[short] (3,1)
      (5,3) to[R=$R_2$,v_<=$U_2$] (5,2) % R2
      to[short] (5,1)
      (3,1) to[short] (4,1)
      (5,1) to[short] (4,1)
      to[short] (4,0)
      to[short] (0,0);
    \end{circuitikz}
    \caption{Conductance d'un circuit parallèle.}
  \end{center}
\end{figure} 

R1 = 50 \ohm R2 = 30 \ohm

Pour ce circuit la conductance $ C = \frac{1}{R1}+\frac{1}{R2} = \frac{1}{50}+\frac{1}{30} = \frac{8}{150} $
Ce qui nous donne $R_t=\frac{150}{8} = 18,75 \ohm $ 

Ici on voit que la résistance totale de notre circuit parallèle est inférieur à la plus petite des résistances.

Si on calcule $i_1=\frac{U_1}{R_1}=\frac{12}{50}=0,24 \ampere $ $i_2=\frac{U_2}{R_2}=\frac{12}{30}=0,4 \ampere $

Si on compare a la résistance totale $I=\frac{12}{18,75}=0,64 \ampere$ nos deux intensités sont bien égales. 

\section{la Puissance}

La puissance P est exprimée en \watt elle se calcule à l'aide de la tension et de l'intensité utilisé par le circuit $P=U*I$.

\begin{figure}[h!]
  \begin{center}    
    \begin{circuitikz}
      \draw (0,0)
      to[V,v=$U_g$] (0,2) % The voltage source
      to[R=$R_i$,v_<=$U_i$] (2,2) % R1
      to[R=$R$,v_<=$U$] (2,0) % R2
      to[short] (0,0);
    \end{circuitikz}
    \caption{Voici mon radiateur.}
  \end{center}
\end{figure}

$R_i$ représente la résistance interne du générateur.    

$R_i=1 \ohm$ et $R = 20 \ohm$

Pour obtenir la puissace fournie par le générateur $P_f$ il faut calculer la somme des résistances $R_t= \sum R = R_i+R=1+20=21 \ohm$

avec $U=R \times I$ on peut déduire que $P=U \times I=R \times I^2=\frac{U^2}{R}$ 

on peut donc déduire directement la puissance fournie par le générateur en utilisant $P_f=\frac{U_g^2}{R}=\frac{12^2}{21}=6,857 \watt$

La puissance utile $P_u$ représente la puissance réellement utilisée est la puissance utilisé par la résistance. Pour cela il faut connaitre le courant utilisé par le circuit $I=\frac{U}{R_t}=\frac{12}{21}=0,571 \ampere$

ce qui nous donne $P_u =R \times I^2=20 \times 0,571^2=6,520 \watt$

on peut donc déduire le rendement de notre résistance $R=\frac{P_u}{P_f}=\frac{6,520}{6,857}=0,985$

\pagebreak

\section{Le condensateur}
Le condensateur est un composant qui se charge lorsqu'il est traversé par du courant jusqu'à ce qu'il ait atteint sa capacité maximal. La capicé d'un condensateur est exprimée en Farad (\farad).

Il est composé de lamelle et d'un espace vide entre elle
\begin{figure}[h!]
  \begin{center}
    \begin{circuitikz}
        \draw (0,0) 
        to node[below,pos=2]{$+q$} node[above,pos=2]{$+$} (0.5,0) 
        to [C, l^=$C$] (2.5,0) 
        to node[below,pos=-1]{$-q$} node[above,pos=-1]{$-$} (3,0) ;
    \end{circuitikz}
    \caption{Schéma d'un condensateur.}
  \end{center}
\end{figure}

Lorsque le courant $i_1$ traverse un condensateur les electrons s'accumulent sur lamelle représenté par $-q$ tandis que l'autre lamelle elle se débarrassent de ses électrons. Au fur et à mesure le courant décroit jusqu'à être nul. À ce moment notre condensateur à atteint sa pleine capacité. il va donc falloir le décharger pour pouvoir faire passer un nouveau courant.

\begin{figure}[h!]
  \begin{center}
    \begin{circuitikz}
        \draw (0,0)
        to[V,v=$U_g$] (0,2) % The voltage source
        to[short] (2,2)
        to[lamp] (4,2)
        to[C,v_<=$U_c$,i^>=$i_c$,l^=$C$] (4,0)
	to[short] (0,0)
	(2,2) to[short] (2,1); 
    \end{circuitikz}
    \caption{Circuit de charge.}
  \end{center}
\end{figure} 

Durant la charge la lampe va s'allumer puis l'intensité dans le circuit va décroite au fur et a mesure que le condensateur se charge.
Jusqu'à ce qu'elle s'éteigne une fois le condensateur complètement chargé. La tension du condensateur est alors maximal voir Figure~\ref{courbeCondensateur} p~\pageref{courbeCondensateur}
\begin{figure}[h!]
  \begin{center}
    \begin{circuitikz}
        \draw (0,0)
        to[V,v=$U_g$] (0,2) % The voltage source
        to[short] (2,2)
        to[lamp] (4,2)
        to[C,v_<=$U_c$,i_<=$i_c$,l^=$C$] (4,0)
        to[short] (2,0)
	(0,0) to[short] (1,0)
        (2,2) to[short] (2,0);
    \end{circuitikz}
    \caption{Circuit de décharge.}
  \end{center}
\end{figure}

Lors de la décharge on s'aperçoit que le courant est inversé voir Figure~\ref{courbeCondensateur} p~\pageref{courbeCondensateur}. Le phénomène est dû au fait que les électrons sont attirés par l'autre lamelle. lors de leur parcourt du circuit ils vont donc allumer la LED jusqu'à temps que le condensateur soit vide. 

Durant la décharge on observe que la tension du condensateur diminue Figure~\ref{courbeCondensateur} p~\pageref{courbeCondensateur}.
\begin{figure}
   \begin{center}
      \begin{tikzpicture}[xmin=0,xmax=16,ymin=-5,ymax=5]
   	  \axes
   	  %\draw [very thin, gray] (0,0) grid[step=0.2] (4.4,4.2);
   	  \draw[color=red] [domain=0:8, smooth] plot (\x,{4/(\x+1)-0.4});
   	  \draw[color=blue] [domain=0:8, smooth] plot (\x,{(4/(-\x-1)+4)});
          \draw[color=blue] [domain=8:16, smooth] plot (\x,{4/(\x-7)-0.4});
          \draw[color=red] [domain=8:16, smooth] plot (\x,{(4/(-\x+7)+0.4)});
	  \draw[color=blue] (0,5) node [right] {$U(\volt)$};
	  \draw[color=red] (0,5) node [left] {$I(\ampere)$};
   	  \draw (16,0) node [below] {t};
   	  \draw (0,0) node [below left] {O};
   	  \fill[color=green!70, opacity=0.5] (0,-0.1) rectangle (8,0.1);
   	  \draw (4,0) node [below] {$charge$};
	  \fill[color=yellow!70, opacity=0.5] (8,-0.1) rectangle (16,0.1);
          \draw (12,0) node [below] {$decharge$};
       \end{tikzpicture}
       \caption{Courbe d'intensité de charge}
       \label{courbeCondensateur}
   \end{center}
\end{figure} 





\end{document}
