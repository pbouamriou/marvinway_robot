\documentclass[10pt,a4paper]{article}
\usepackage[utf8]{inputenc}
\usepackage[french]{babel}
\usepackage[T1]{fontenc}
\usepackage[left=2cm,right=2cm,top=2cm,bottom=2cm]{geometry}

 
\begin{document}{\bf Le Besoin de Marvin} 

\section{Comment faire pour donner vie à Marvin}

\begin{itemize}
\item Mangez => Batterie
\item Digérer => Carte Puissance
\item Tenir en équilibre => Carte Moteur
\item Découvrir son environment => Carte Moteur
\item être intelligent => BeagleBone
\end{itemize}

C'est moi dany la star en électronique.
Bah non tu t'appelles philippe maintenant


\section{Détails Technique}

    \begin{itemize}
    	\item Trouver une batterie pas trop lourdes (Batterie LiPo-Succion)
    	\item Trouver ou la positionner sur le schéma mécanique
    	\item Gestion de la décharge de la batterie (LiPo-Succion Algo)
    	\item Carte de puissance: Son rôle
    	\begin{itemize}
        	\item Charger une batterie
        	\item Délivrer les différentes tensions aux différentes cartes.
        	\item Faire les mesures de courant
        	\item Gère le fait de pouvoir alimenter => v5 (surtout si c'est fait par un noob de l'électronique
    	\end{itemize}
    	\item Carte Moteurs : Son rôle
    	\begin{itemize}
        	\item Piloter les moteurs (Pont en H, piloté par la Teensy)
        	\item Filtrage de Kaalman
        	\item Controle les capteurs (Position Angulaire)
        	\item Execute les commandes envoyé par le cerveau(Vitesse RoueG, Vitesse RoueD)
        	\item Récupérer la position des roues (Optionnel parce que plus cher) => v4 (Parce que la V3 marche pas)
        	\item Autre : Trouver la réponse à la vie l'univers et tout ! => Revenez dans quelques Milliard d'années
    	\end{itemize}
    	\item  Beagleboned+Shield
    	\begin{itemize}
        	\item Envoi des consignes (ordre moteur), supervision des données remontées par les différentes cartes (gyro, charge de la batterie),
        	\item Réception des ordres d'une (Wiimote, Smartphone...)
        	\item Interface Web de Debug (
            	\begin{itemize}
            		\item Voir l'état des capteurs en live
            		\item Voir l'état des batteries
            		\item Voir l'équilibre de Marvin
            		\item Autre données intérréssantes
            		\item  Commande de Base avancer, reculer, Standby
            		\item Voir la Webcam(v3)
            	\end{itemize}
        	\item Module sonore pour faire Coin ! et autre je suis déprimé (v42)
        	\item Pilotage de deux cerveau moteur(Panen-Tilt)(v4)
        	\item Pilotage de la caméra(v3)
  	\end{itemize}
    \end{itemize}
    Détails: On alimente ls cartes moteurs et la Shield avec la carte de puissance. La beaglebone sera alimenté par la Shield
\section{Les centres d'intérets de chacun}

\begin{itemize}
\item Guy Super Canard: 
    \begin{itemize}
    \item Faire du Linux
    \item Haut Niveau (Vers l'infini et l'au delà)
    \item Il est Open (Source ?)
    \end{itemize}
\item Dany:
    \begin{itemize}
    \item bas niveau (c'est petit)
    \item Partie Bootloader (Donne des coups de pied pour démarrer)
    \item => Passe par le BUS CAN Réalisé côté Shield Le But étant de pouvoir flasher rapidement les autres carte avec une interface unique
    \end{itemize}
\item Philippe:
    \begin{itemize}
    \item Les Problèmes Et-les-Chroniques de Marvin Analogique et Numéraire (Voir Calendaire) On est pas sur le calendrier Maya il est fini
    \item Bas niveau voir proche du centre de la terre
    \end{itemize}
\item  Nicolas:
    \begin{itemize}
    \item Les problèmes ÉleChronique de Marvin (On va faire sauter les fusibles à Dany !)
    \item Filtrage de Kaalman (Mais besoin d'aide ça sembe compliqué il y a des maths)
    \end{itemize}
\end{itemize}

\section{Kifékoi}
\begin{itemize}
\item Dany : 
    \begin{itemize}
   \item Carte extension Beaglebone(Shield)
   \end{itemize}
\item Guy:
   \begin{itemize}
   \item Mettre en place un linux => BeagleBone
   \item installer un Serveur NodeJs(ou autre si tu trouve mieux) => Sur le linux Beaglebone
   \item on veut des sous Pour me soudoyer il faut minimun 20 euros et 1 Bière
   \end{itemize}
\item Philippe:
    \begin{itemize}
    \item Concevoir la carte de puissance (ça marchera du premier coup biensûr)
    \item concevoir la carte moteur
    \end{itemize}
\item Nicolas:
    \begin{itemize}
    \item Concevoir la carte de puissance et la tester (ça marchera du premier coup biensûr)
    \item Concevoir la carte Moteur
    \end{itemize}
\end{itemize}
\end{document}
