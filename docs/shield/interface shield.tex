\documentclass[11pt]{article}
\usepackage[utf8]{inputenc}
\usepackage[T1]{fontenc}
\usepackage[francais]{babel}
\usepackage{bytefield}

\begin{document}

\title{Interface MarvinShield}

\section{Introduction}

Le FPGA est piloté en SPI.


\section{Commandes}

Il existe deux types de commandes :
\begin{itemize}
  \item une commande de lecture d'une zone mémoire,
  \item une commande d'écriture d'une zone mémoire. 
\end{itemize}

La figure suivante présente le chronogramme d'une phase de lecture.

La figure suivante présente le chronogramme d'une phase d'écriture.

\subsection{Organisation mémoire}

La carte MarvinShield organise la mémoire de la manière suivante : \\

\begin{bytefield}[endianness=big, bitwidth=1em, leftcurly=., leftcurlyspace=0pt]{32}
\bitheader{0,7,8,15,16,23,24,31} \\
\begin{leftwordgroup}{00h}
\bitbox{32}{FPGA Magic ID}
\end{leftwordgroup} \\
\begin{rightwordgroup}{Version}
\begin{leftwordgroup}{04h}
\bitbox{8}{Reserved} \bitbox{8}{Majeur} \bitbox{8}{Mineur} 
& \bitbox{8}{Revision}
\end{leftwordgroup}
\end{rightwordgroup} \\
\begin{leftwordgroup}{08h}
\wordbox[tlr]{1}{}
\end{leftwordgroup} \\
\wordbox[lr]{2}{FPGA Desc} 	\\
\wordbox[blr]{1}{} \\
\begin{rightwordgroup}{Capabilities}	
\begin{leftwordgroup}{18h}
\bitbox{16}{Capability (1)} \bitbox{16}{Capability (2)}
\end{leftwordgroup} \\
\wordbox{6}{$\cdots$} \\
\begin{leftwordgroup}{38h}
\bitbox{16}{Capability (15)} \bitbox{16}{Capability (16)}
\end{leftwordgroup}
\end{rightwordgroup} \\
\begin{leftwordgroup}{3Ch}
\wordbox{1}{First device address}
\end{leftwordgroup}
\end{bytefield}

Tous les champs
Le champ "FPGA Magic ID" est toujours positionné par la shield  

\subsection{Capacités du FPGA}

\begin{bytefield}[endianness=little,bitwidth=2.2em]{16}
\bitheader{0-15} \\ 
\bitbox{3}{device ID} \bitbox{3}{device version} \bitbox{10}{Offset in memory} \\
\end{bytefield}

La liste des différents périphériques sont :
\begin{itemize}
\item PWM, valeur 01h
\item ADC, valeur 02h
\item GPIO, valeur 03h
\end{itemize}

Le champ "Offset in memory" défini l'adressage du périphérique à partir de l'adresse 3Ch.


\end{document}
